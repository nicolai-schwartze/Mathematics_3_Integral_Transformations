\documentclass[./\jobname.tex]{subfiles}
\begin{document}

\chapter{Z-Transformation}

The Z-transformation, also called the discrete Laplace-transformation, is used for discrete sequences. The Z-transformation can be applied to solve difference equations. 

\begin{equation}
	\mathcal{Z} \left( f\left[ n \right] \right) = F(z)
\end{equation}

\section{Z-Transformation Sum}

The Z-transformation is calculated by evaluating the following sum:

\begin{equation}
	Z(f[n]) = \sum_{n=0}^{\infty} f[n] z^{-n}
\end{equation}

\textbf{Insertion: } Geometric Series

The geometric series is especially helpfule to solve many Z-transformations. It is defined as 

\begin{equation}
	\sum_{i=0}^{\infty} x^i = \frac{1}{1-x}
\end{equation}

However, this is only applicable if $|x| < 1$ holds. This is assumed for all further statements. 

\section{Inverse Z-Transformation}

The inverse Z-transformation can often be solved by a partial fraction decomposition, just like the inverse Laplace-transformation. Every partial fraction is separately transformed. 

\begin{equation}
	\mathcal{Z}^{-1}\left( F(z) \right) = f[n]
\end{equation}


\section{Collection of Common Z-Transformation}

\begin{table}[H]
	\centering
	\noindent\adjustbox{max width=\linewidth}{
		\begin{tabular}{|c|c|}
			\hline
			\rowcolor[HTML]{\farbeTabA}
			Series $f[n]$ & Z-Transformation $F(z)$ \\ \hline
			$1 \text{ or } u[n]$ & $\frac{z}{z-1}$ \\ \hline
			$\delta[n]$ & $1$ \\ \hline
			$\delta[n-1]$ & $\frac{1}{z}$ \\ \hline
			$\delta[n-2]$ & $\frac{1}{z^2}$ \\ \hline
			$\delta[n-k]$ & $\frac{1}{z^k}$ \\ \hline
			$a^n$ & $\frac{z}{z - a}$ \\ \hline
			$e^n$ & $\frac{z}{z - e}$ \\ \hline
			$(-a)^n$ & $\frac{z}{z + a}$ \\ \hline
			$sin[a \cdot n]$ & $\frac{z \cdot sin(a)}{z^2 - 2 z \cdot cos(a) + 1}$ \\ \hline
			$cos[a \cdot n]$ & $\frac{z^2 - z \cdot cos(a)}{z^2 - 2 z \cdot cos(a) + 1}$ \\ \hline
			$n$ & $\frac{z}{(z - 1)^2}$ \\ \hline
			$n^2$ & $\frac{z \cdot (z + 1)}{(z - 1)^3}$ \\ \hline
			$n^3$ & $\frac{z \cdot (z^2 + 4z + 1)}{(z - 1)^4}$ \\ \hline
			$n^4$ & $\frac{z \cdot (z^3 + 11z^2 + 11z + 1)}{(z - 1)^5}$ \\ \hline
			$sinh[a \cdot n]$ & $\frac{z \cdot sinh(a)}{z^2 - 2 z \cdot cosh(a) + 1}$ \\ \hline
			$cosh[a \cdot n]$ & $\frac{z^2 -z \cdot cosh(a)}{z^2 - 2 z \cdot cosh(a) + 1}$ \\ \hline
			$sin\left[\frac{\pi}{2} \cdot n\right]$ & $\frac{z}{z^2 + 1}$ \\ \hline
			$cos\left[\frac{\pi}{2} \cdot n\right]$ & $\frac{z^2}{z^2 + 1}$ \\ \hline
		\end{tabular}
	}
\end{table}

\section{Properties}

Similary to the Laplace transformation, the Z-transformation also has some important properties that can be exploited to solve most of the upcoming problems. 

\subsection{Linearity}
The Z-transformation is a linear transformation which means that the linearity condition holds. Summations can be split and multiplicative constants can be pulled infront of the sum. 

\begin{equation}
	\mathcal{Z} \left( a \cdot f[n]  \pm b \cdot g[n] \right) = a F(f[n]) \pm b F(f[n])
\end{equation}

\subsection{Multiplication Theorem}

If a sequence is multiplied with n in the time domain, the Z-transformation must be differentiated and multiplied with $-z$. 

\begin{equation}
	\mathcal{Z} (n \cdot f[n]) = -z \cdot \frac{dF}{dz}
\end{equation}

\subsection{Time Shift Right}

Similarly as in continuous function, the time shift in positive time direction can be acomplished by subtraction

\begin{equation}
	f[n] \rightarrow f[n-1]
\end{equation}

The transformation of such a function can be done by:

\begin{equation}
	\begin{split}
		\mathcal{Z} (f[n-1]) = & f[-1] + \frac{1}{z} \cdot \mathcal{Z}(f[n]) \\
		\mathcal{Z} (f[n-2]) = & f[-2] + \frac{f[-1]}{z} + \frac{1}{z^2} \mathcal{Z}(f[n]) \\
		\mathcal{Z} f([n-3]) = & f[-3] + \frac{f[-2]}{z} + \frac{f[-2]}{z^2} + \frac{1}{z^3} \mathcal{Z}(f[n])
	\end{split}
\end{equation}

or for any shift in positive time direction by $a$ time steps: 

\begin{equation}
	\mathcal{Z}(f[n-a]) \left( \sum_{i=0}^{i=(a-1)} \frac{f[-a + i]}{z^i} \right) + \frac{\mathcal{Z}(f[n])}{z^a}
\end{equation}

When considering the inverse Z-transformation, a multiplication with $\frac{1}{z}$ in the z-domain, must be shifted via a Heaviside function in the discrete time domain. An example could be: 

\begin{equation}
	\mathcal{Z}^{-1} \left( \frac{z}{z-4} \cdot \frac{1}{z} \right) = 4^n \cdot u[n-1]
\end{equation}

\subsection{Time Shift Left}

Contrary to the time shift in positive direction, a shift into the negative time direction is done by an addition: 

\begin{equation}
	f[n] \rightarrow f[n + 1]
\end{equation}

The transformation can be expressed as

\begin{equation}
	\begin{split}
		\mathcal{Z}(f[n+1]) = & z \cdot (\mathcal{Z}(f[n]) - f[0]) \\
		\mathcal{Z}(f[n+2]) = & z^2 \cdot \left( \mathcal{Z}(f[n]) - f[0] - \frac{f[1]}{z} \right) \\
		\mathcal{Z}(f[n+3]) = & z^3 \cdot \left( \mathcal{Z}(f[n]) - f[0] - \frac{f[1]}{z} - \frac{f[2]}{z^2}\right)
	\end{split}
\end{equation}

or a shift by any arbitrary constant $a$

\begin{equation}
	\mathcal{Z}(f[n+a]) = z^a \cdot \left( \mathcal{Z}(f[n]) - \sum_{j=0}^{a-1} \frac{f[j]}{z^j}\right)
\end{equation}

\subsection{Convolution}

A convolution in the discrete time domain is defined as 

\begin{equation}
	f[n] * g[n] = \sum_{k=0}^{n} f[k] \cdot g[n-k]
\end{equation}

The Z-transformation for the convolution of two signals can be calculated by

\begin{equation}
	\begin{split}
		\mathcal{Z}(f[n] * g[n]) = & \mathcal{Z}(f[n]) \cdot \mathcal{Z}(g[n]) \\
		\mathcal{Z}^{-1}(F(z) \cdot G(z)) = & f[n] * g[n]
	\end{split}
\end{equation}

\end{document}
