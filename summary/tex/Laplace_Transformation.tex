\documentclass[./\jobname.tex]{subfiles}
\begin{document}

\chapter{Laplace Transformation}

The Laplace transformation can be used to solve differential equations. For that purpose, the Laplace transformation converts a function from the time domain into the s-domain.
The notation is 

\begin{equation}
	f(t) \text{ } \laplace \text{ } \mathcal{L}(f(t)) = F(s)
\end{equation}

\section{Laplace Integral}
The Laplace transformation is defined as 
\begin{equation}
	\mathcal{L} \left( f(t) \right) = \int_{0}^{\infty} f(t) \cdot e^{-st} dt
\end{equation}

where $s$ is a complex number $s = \delta + i \omega$. This transformation only works on functions $f(t)$ that do not grow faster than $e^t$. There are many standard transforamtions as seen in the following table. 

\section{Inverse Laplace Transformation}
The inverse Laplace transformation brings the function brack from the s-domain into the time domain. This is noted like 
\begin{equation}
	\mathcal{L}^{-1} \left(F(s)\right) = f(t)
\end{equation}

In most cases the inverse transformation can be done by a partial fraction decomposition. Therefore, the function $f(s)$ is written as a sum of fractions, where each fraction can be separately transformed. 

\section{Collection of Common Laplace-Transformation}

\begin{table}[H]
	\centering
	\noindent\adjustbox{max width=\linewidth}{
		\begin{tabular}{|c|c|}
			\hline
			\rowcolor[HTML]{\farbeTabA}
			Function $f(t)$ & Transformation $F(s)$  \\ \hline
			$1$ & $\frac{1}{s}$  \\ \hline
			$t$ & $\frac{1}{s^2}$  \\ \hline
			$t^n, \text{ } n \in \mathbb{N}$ & $\frac{n!}{s^{n+1}}$  \\ \hline
			$e^{\pm at}$ & $\frac{1}{s \mp a}$  \\ \hline
			$t \cdot e^{\pm at}$ & $\frac{1}{(s \mp a)^2}$  \\ \hline
			$t^n \cdot e^{\pm at}$ & $\frac{n!}{(s \mp a)^{n + 1}}$  \\ \hline
			$u(t -a)$ & $\frac{1}{s} e^{-as}$  \\ \hline
			$f(t-a) \cdot u(t -a)$ & $\mathcal{L}(f(t)) \cdot e^{-as}$  \\ \hline
			$\delta(t -a)$ & $e^{-as}$  \\ \hline
			$\sqrt{t}$ & $\frac{1}{2s} \sqrt{\frac{\pi}{s}}$  \\ \hline
			$\frac{1}{\sqrt{t}}$ & $\sqrt{\frac{\pi}{s}}$  \\ \hline
			$\sqrt{t} \cdot e^{at}$ & $\frac{\sqrt{\pi}}{2(s -a)\sqrt{s - a}}$  \\ \hline
			$\frac{1}{\sqrt{t}} \cdot e^{at}$ & $\frac{\sqrt{\pi}}{\sqrt{s - a}}$ \\ \hline
			$sin(\omega t)$ & $\frac{\omega}{s^2 + \omega^2}$ \\ \hline
			$cos(\omega t)$ & $\frac{s}{s^2 + \omega^2}$ \\ \hline
			$t \cdot sin(\omega t)$ & $\frac{2 \omega s}{(s^2 + \omega^2)^2}$ \\ \hline
			$t \cdot cos(\omega t)$ & $\frac{s^2 - \omega^2}{(s^2 + \omega^2)^2}$ \\ \hline
			$t^n \cdot sin(\omega t), \text{ } n \in \mathbb{N}$ & $\frac{i \cdot n!}{2} \left( \frac{1}{(s + i \omega)^{n + 1}} - \frac{1}{(s - i \omega)^{n + 1}}\right)$ \\ \hline
			$t^n \cdot cos(\omega t), \text{ } n \in \mathbb{N}$ & $\frac{n!}{2} \left( \frac{1}{(s + i \omega)^{n + 1}} + \frac{1}{(s - i \omega)^{n + 1}}\right)$ \\ \hline
			$sinh(\omega t)$ & $\frac{\omega}{s^2 - \omega^2}$ \\ \hline
			$cosh(\omega t)$ & $\frac{s}{s^2 - \omega^2}$ \\ \hline
			$t \cdot sinh(\omega t)$ & $\frac{2 \omega s}{(s^2 - \omega^2)^2}$ \\ \hline
			$t \cdot cosh(\omega t)$ & $\frac{s^2 + \omega^2}{(s^2 - \omega^2)^2}$ \\ \hline
			$t^n \cdot sinh(\omega t), \text{ } n \in \mathbb{N}$ & $ \frac{n!}{2} \left( \frac{1}{(s - \omega)^{n + 1}} - \frac{1}{(s + \omega)^{n+1}} \right)$ \\ \hline
			$t^n \cdot cosh(\omega t), \text{ } n \in \mathbb{N}$ & $ \frac{n!}{2} \left( \frac{1}{(s - \omega)^{n + 1}} + \frac{1}{(s + \omega)^{n+1}} \right)$ \\ \hline
			$e^{at} sin(\omega t)$ & $\frac{\omega}{(s - a)^2 + \omega^2}$ \\ \hline
			$e^{at} cos(\omega t)$ & $\frac{s - a}{(s - a)^2 + \omega^2}$ \\ \hline
			$e^{at} sinh(\omega t)$ & $\frac{\omega}{(s - a)^2 - \omega^2}$ \\ \hline
			$e^{at} cosh(\omega t)$ & $\frac{s - a}{(s - a)^2 - \omega^2}$ \\ \hline
			$sin(\omega t)^2$ & $\frac{2 \omega^2}{s(s^2 + 4\omega^2)}$ \\ \hline
			$cos(\omega t)^2$ & $\frac{s^2 + 2\omega^2}{s(s^2 + 4\omega^2)}$ \\ \hline
			$sinh(\omega t)^2$ & $\frac{2 \omega^2}{s(s^2 - 4\omega^2)}$ \\ \hline
			$cosh(\omega t)^2$ & $\frac{s^2 - 2\omega^2}{s(s^2 - 4\omega^2)}$ \\ \hline
		\end{tabular}
	}
\end{table}

\newpage

\section{Properties}

The Laplace transformation has some helpfule properties, that can be leveraged to solve a problem. 

\subsection{Linearity}

The Laplace transformation is a linear operation. Therefore, the linearity condition holds. The summation of multiple functions can be split and transformed seperatly. 

\begin{equation}
	\mathcal{L} \left(a \cdot f(t) \pm b \cdot g(t)\right) = a \mathcal{L} \left( f(t) \right) \pm b \mathcal{L} \left( g(t) \right)
\end{equation}

\subsection{Derivative}
\label{chap:derivative}

Differentiating a function in the time domain corresponds to a multiplication with $s$ in the s-domain. 

\begin{equation}
	\begin{split}
		& \mathcal{L} \left( f'(t) \right) = s \cdot \mathcal{L} \left( f(t) \right) - f(0) \\
		& \mathcal{L} \left( f''(t) \right) =  s^2 \cdot \mathcal{L} \left( f(t) \right) - s \cdot f(0) - f'(0) \\
		& \mathcal{L} \left( f^{(n)}(t) \right) = s^n F(s) - \sum_{k=1}^{n} s^{n-k} f^{(k-1)}(0)
	\end{split}
\end{equation}

\subsection{Integration}

Similary, the integration in of the function in the time domain corresponds with a multiplication of $\frac{1}{s}$ in the s-domain. This relation is especially helpfule for inverse Laplace transformations.

\begin{equation}
	\mathcal{L}\left( \int_{\tau=0}^{t} f(\tau) d\tau \right) = \frac{1}{s} \mathcal{L} \left( f(t) \right)
\end{equation}

The diagram below visualizes the correspondence in a different way. 

\begin{figure}[H]
	\centering
	\noindent\adjustbox{max width=\linewidth}{
		\tikzset{every picture/.style={line width=0.75pt}} %set default line width to 0.75pt        
		
		\begin{tikzpicture}[x=0.75pt,y=0.75pt,yscale=-1,xscale=1]
		%uncomment if require: \path (0,171); %set diagram left start at 0, and has height of 171
		
		
		% Text Node
		\draw (213,31) node    {$f( t)$};
		% Text Node
		\draw (213,131) node    {$g( t)$};
		% Text Node
		\draw (457,31) node    {$F( s)$};
		% Text Node
		\draw (457,131) node    {$G( s)$};
		% Text Node
		\draw (467,58.4) node [anchor=north west][inner sep=0.75pt]    {$\frac{1}{s}$};
		% Text Node
		\draw (314,9.4) node [anchor=north west][inner sep=0.75pt]    {$\mathcal{L}( ...)$};
		% Text Node
		\draw (316,137.4) node [anchor=north west][inner sep=0.75pt]    {$\mathcal{L}( ...)$};
		% Text Node
		\draw (165,56.4) node [anchor=north west][inner sep=0.75pt]    {$\int dt$};
		% Connection
		\draw    (230,31) -- (437.5,31) ;
		\draw [shift={(439.5,31)}, rotate = 180] [color={rgb, 255:red, 0; green, 0; blue, 0 }  ][line width=0.75]    (10.93,-3.29) .. controls (6.95,-1.4) and (3.31,-0.3) .. (0,0) .. controls (3.31,0.3) and (6.95,1.4) .. (10.93,3.29)   ;
		% Connection
		\draw    (213,43) -- (213,117) ;
		\draw [shift={(213,119)}, rotate = 270] [color={rgb, 255:red, 0; green, 0; blue, 0 }  ][line width=0.75]    (10.93,-3.29) .. controls (6.95,-1.4) and (3.31,-0.3) .. (0,0) .. controls (3.31,0.3) and (6.95,1.4) .. (10.93,3.29)   ;
		% Connection
		\draw    (229.5,131) -- (436,131) ;
		\draw [shift={(438,131)}, rotate = 180] [color={rgb, 255:red, 0; green, 0; blue, 0 }  ][line width=0.75]    (10.93,-3.29) .. controls (6.95,-1.4) and (3.31,-0.3) .. (0,0) .. controls (3.31,0.3) and (6.95,1.4) .. (10.93,3.29)   ;
		% Connection
		\draw    (457,43) -- (457,117) ;
		\draw [shift={(457,119)}, rotate = 270] [color={rgb, 255:red, 0; green, 0; blue, 0 }  ][line width=0.75]    (10.93,-3.29) .. controls (6.95,-1.4) and (3.31,-0.3) .. (0,0) .. controls (3.31,0.3) and (6.95,1.4) .. (10.93,3.29)   ;
		
		\end{tikzpicture}
	}
\end{figure}
    

\subsection{Frequency - Shift}

If the function $f(t)$ is multiplied with an exponential function in the time domain, than the exponent $a$ must be subtracted or added in the transformation. The addition or subtraction must be applied to every s in the s-domain. Thus, the function gets shifted in the s-domain which is also the reason for the name of this porperty.

\begin{equation}
	\mathcal{L} \left(f(t) \cdot e^{\pm at}\right) \rightarrow F(s \mp a)
\end{equation} 


\subsection{Time - Shift}

A function can also be shifted in the time domain. One way of shifting a function is to multiply it with the Heaviside function. Strictly speaking, the Heaviside ``distribution'' is not a function. 

\subsubsection{Heaviside Function}

The hHeaviside distrubution is defined as: 

\begin{equation}
	u(t-a) = 
	\begin{cases}
	0 & \text{ for } t < a \\
	1 & \text{ for } t \geq a
	\end{cases}
\end{equation}

The Heaviside function is $0$ for every all time before $t < a$ and $1$ for all time beyond $t \geq a$. 

When a function $f(t)$ is multiplied with the Heaviside function $u(t-a)$, the result $g(t)$ is restricted to the time $t \geq a$. For $t < a$, $g(t) = 0$ and for $t \geq a$, $g(t) = f(t)$. 

A function, that is multiplied with the Heaviside function, can be Laplace transformed in the following way:

\begin{equation}
\begin{split}
\mathcal{L}\left( u(t-a) \right) & = \frac{e^{-a}}{s} \\
\mathcal{L}\left( f(t-a) \cdot u(t-a) \right) & = \mathcal{L}\left( f(t) \right) \cdot e^{-as} \\
\mathcal{L}^{-1}\left( F(s) \cdot e^{-as} \right) & = f(t-a) \cdot u(t-a)
\end{split}
\end{equation}

\subsubsection{Dirac's Deltafunction}

The Dirac's delta function can be thought of as an infinitly high needle at only one distinct place. It is defined as: 

\begin{equation}
	\delta (t-a) = 
	\begin{cases}
	\infty & \text{ for } t = a \\
	0 & \text{ for } t \neq a
	\end{cases}
\end{equation}

The Dirac delta impulse is the derivative of the Heaviside function, which is especially helpfule when using the derivative property from \ref{chap:derivative}. 

\subsection{Convolution}

The convolution of two functions $f(t)$ and $g(t)$ is often denominated by the $*$ operator. The convolution is defined as:

\begin{equation}
	f(t)*g(t) = \int_{\tau = 0}^{t} f(\tau) \cdot g(t-\tau) d\tau
\end{equation}

The special property of the convolution is that the Laplace transformation of two convoluted functions in the time domain is equivalent to the muliplication in the s-domain. 

\begin{equation}
	\mathcal{L} \left( f(t) * g(t) \right) = \mathcal{L} \left( f(t) \right) \cdot \left( g(t) \right)
\end{equation}

\begin{equation}
	\mathcal{L}^{-1} \left( F(s) \cdot G(s) \right) = \mathcal{L}^{-1}( F(s)) * \mathcal{L}^{-1} (G(s))
\end{equation}

\subsection{Initial Value Theorem}

With the initial value theorem one can very quickly evaluate the function at $t=0$. For this purpose, the limit of the function multiplied with $s$ is taken in the s-domain. For fractions it might be helpfule to apply the limit rule of de l'Hospital. 

\begin{equation}
	f(0) = \underset{s \rightarrow \infty}{lim} F(s) \cdot s
\end{equation}

\subsection{Final Value Theorem}

Similar to the initial value theorem, the final value theorem can evaluate the limit of the function at $t \rightarrow \infty$. 

\begin{equation}
	\underset{t \rightarrow \infty}{lim} f(t) = \underset{s \rightarrow 0}{F(s) \cdot s}
\end{equation}

\subsection{Frequency Domain Derivative}
This theorem is sometimes also called the muliplication theorem. Here $n \in \mathbb{N}$ not only denotes an natural exponent but also the number of derivatives. 

\begin{equation}
	\mathcal{L} \left( t^n \cdot f(t) \right) = (-1)^n \cdot \frac{d^{(n)}}{ds^{(n)}} F(s)
\end{equation}


\subsection{Frequency Domain Integral}

Similary, this theorem is sometimes also called the division theorem. To prevent a confusion between the integration variable s and the Laplace-s, the integration variable is denoted as $\tilde{s}$.

\begin{equation}
\mathcal{L} \left(\frac{f(t)}{t}\right) = \int_{\tilde{s} = s}^{\infty} F(\tilde{s}) d\tilde{s}
\end{equation}

\newpage

\end{document}
